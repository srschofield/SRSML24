\documentclass[11pt]{article}
\usepackage[utf8]{inputenc}
\usepackage[T1]{fontenc}
\usepackage{geometry}
\usepackage{hyperref}
\geometry{margin=1in}

\usepackage{longtable}
\usepackage{array}
\newcolumntype{L}[1]{>{\raggedright\arraybackslash}p{#1}}

\title{SRSML24: STM Machine Learning Module}
\author{Steven R. Schofield}
\date{\today}

\begin{document}

\maketitle

\section*{Overview}
This module provides tools for machine learning analysis of scanning tunnelling microscopy (STM) data, including autoencoder models, clustering tools, and STM-specific preprocessing.

\section*{Getting the Code}
Clone the repository from GitHub:

\begin{verbatim}
git clone https://github.com/srschofield/SRSML24.git
\end{verbatim}

\section*{Installation}
It is recommended to create a clean Python environment using \texttt{conda}. The following steps assume you are working on a macOS system with Apple Silicon:

\begin{verbatim}
# create and activate environment
conda create --name srsml24 python=3.8 -y
conda activate srsml24

# install packages
pip install -r requirements-macos.txt
\end{verbatim}

\subsection*{Known Working Configuration}
This module has been tested and is known to work with the following configuration on macOS 15.0.1 (Apple Silicon, M3 Pro chip):

\begin{itemize}
  \item \texttt{python==3.8}
  \item \texttt{tensorflow-macos==2.16.2}
  \item \texttt{tensorflow-metal==1.1.0}
  \item \texttt{numpy==1.24.3}
  \item \texttt{pandas==1.5.3}
  \item \texttt{matplotlib==3.7.1}
  \item \texttt{scikit-learn==1.3.0}
  \item \texttt{scipy==1.10.1}
  \item \texttt{opencv-python==4.8.1.78}
  \item \texttt{Pillow==9.5.0}
  \item \texttt{joblib==1.3.2}
  \item \texttt{jupyter==1.0.0}
  \item \texttt{ipykernel==6.29.3}
  \item \texttt{keras-core==0.1.6}
  \item \texttt{spiepy==0.1.6}
  \item \texttt{access2thematrix==0.1.3}
\end{itemize}

These packages can be installed using the \texttt{requirements-macos.txt} file. The Python version is critical: other versions may cause compatibility issues with TensorFlow or other packages on Apple Silicon.

\section*{Python Files}
\begin{itemize}
  \item \texttt{data\_prep.py} – Functions for data preparation, including slicing STM images into windows and saving them in efficient formats.
  \item \texttt{model.py} – Defines convolutional autoencoder and UNET-style models.
  \item \texttt{utils.py} – Utility functions for loading/saving models, feature arrays, and results.
\end{itemize}

\section*{Example Data and Scripts}
\begin{itemize}
  \item \texttt{example\_data/} – Example STM data and latent feature arrays.
  \item \texttt{examples/} – Scripts demonstrating usage of the module, including training and inference workflows.
\end{itemize}

\section*{License}
This work is licensed under the Creative Commons Attribution-NonCommercial-ShareAlike 4.0 International License (CC BY-NC-SA 4.0).  
You are free to:
\begin{itemize}
  \item \textbf{Share} — copy and redistribute the material in any medium or format
  \item \textbf{Adapt} — remix, transform, and build upon the material
\end{itemize}
Under the following terms:
\begin{itemize}
  \item \textbf{Attribution} — You must give appropriate credit, provide a link to the license, and indicate if changes were made.
  \item \textbf{NonCommercial} — You may not use the material for commercial purposes.
  \item \textbf{ShareAlike} — If you remix, transform, or build upon the material, you must distribute your contributions under the same license.
\end{itemize}

To view a copy of this license, visit:  
\href{https://creativecommons.org/licenses/by-nc-sa/4.0/}{https://creativecommons.org/licenses/by-nc-sa/4.0/}

\section*{Parameter Summary}

\begin{longtable}{L{0.3\textwidth} L{0.65\textwidth}}
\textbf{Parameter} & \textbf{Description} \\
\hline
\endfirsthead

\textbf{Parameter} & \textbf{Description} \\
\hline
\endhead

\multicolumn{2}{l}{\textbf{General}} \\
\texttt{job\_name} & Label for the run, it will be the folder name for output. \\
\texttt{verbose} & If \texttt{True}, enables more detailed print output. \\

\hline
\multicolumn{2}{l}{\textbf{Matrix data file processing}} \\ 
\texttt{flatten\_method} & Method used to flatten STM images before analysis. Options are `none', `iterate\_mask', `poly\_xy'.\\
\texttt{pixel\_density} & All images will be converted to this pixel density (px/nm). \\
\texttt{pixel\_ratio} & Images that have ratio of fast/slow scan direction less than this will be discared. Setting to 1 means only complete (square) images are kept.\\
\texttt{data\_scaling} & Multiplicative factor for z-height data. Setting to 1.e9 means that the range 0--1 (used for training) corresponds to 1 nm.\\

\hline
\multicolumn{2}{l}{\textbf{Window generation}} \\ 
\texttt{window\_size} & Side length of square image windows (in pixels). \\
\texttt{window\_pitch} & Spacing between adjacent windows during tiling. \\

\hline
\multicolumn{2}{l}{\textbf{Data saving}} \\

\multicolumn{2}{l}{(Should remain defaults but options can be useful for examining data manually.)}\\
\texttt{save\_windows} & If \texttt{True}, saves image windows as \texttt{.npy} files (True).\\
\texttt{together} & If \texttt{True}, saves windows per image in a single file (True). \\
\texttt{save\_jpg} & If \texttt{True}, saves full STM images as JPGs (False).\\ 
\texttt{collate} & If \texttt{True}, flattens directory structure into one folder. (False). \\
\texttt{save\_window\_jpgs} & If \texttt{True}, saves image windows as JPGs. (False)\\

\hline
\multicolumn{2}{l}{\textbf{Autoencoder}} \\ 
\texttt{model\_name} & Label used to save and load the trained autoencoder model. \\
\texttt{batch\_size} & Number of windows per training batch. \\
\texttt{buffer\_size} & Size of shuffle buffer. \\
\texttt{learning\_rate} & Learning rate for the optimizer. \\
\texttt{epochs} & Number of training epochs. \\

\hline
\multicolumn{2}{l}{\textbf{Clustering}} \\ 
\texttt{cluster\_model\_name} & Name used when saving the clustering model. \\
\texttt{cluster\_batch\_size} & Number of latent vectors per clustering batch. \\
\texttt{cluster\_buffer\_size} & Size of buffer for clustering shuffle. \\
\texttt{num\_clusters} & Number of clusters to form using KMeans. \\
\texttt{n\_init} & Number of initializations for KMeans. \\
\texttt{max\_iter} & Max iterations for KMeans convergence. \\
\texttt{reassignment\_ratio} & Fraction of centroids reassigned each step. \\

\hline
\multicolumn{2}{l}{\textbf{Image prediction}} \\ 
\texttt{predict\_window\_pitch} & Window spacing during prediction step. \\
\texttt{mtrx\_train\_data\_limit} & Max number of training MTRX files to use. \\
\texttt{mtrx\_test\_data\_limit} & Max number of validation MTRX files to use. \\
\texttt{train\_data\_limit} & Limit on number of training windows. \\
\texttt{test\_data\_limit} & Limit on number of validation windows. \\
\end{longtable}

\end{document}
